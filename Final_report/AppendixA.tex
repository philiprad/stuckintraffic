\newpage
\section*{Source code software appendix} % Main appendix title

%\label{AppendixA} % For referencing this appendix elsewhere, use \ref{AppendixA}

Stuck in traffic source code: Simulation view % This is for the header on each page - perhaps a shortened title

\lstset{language=Java, framexleftmargin=0pt, framexrightmargin=10pt, frame=single, breaklines=true, }          % Set your language (you can change the language for each code-block optionally) 

\begin{lstlisting}[numbers=left, numberstyle=\small, numbersep=8pt,  framexleftmargin=1pt, framexrightmargin=10pt ]  % Start your code-block
/*
 * @author  Maxim Vasilishin
 * @version 1.0
 */
package gui;

import gui.RoadEditorView.ExitListener;
import gui.RoadEditorView.MainMenuListener;
//import gui.RoadEditorView.OpenListener;
import java.awt.BorderLayout;
import java.awt.Button;
import java.awt.Color;
import java.awt.Component;
import java.awt.Dimension;
import java.awt.EventQueue;
import java.awt.Font;
import java.awt.event.ActionEvent;
import java.awt.event.ActionListener;
import java.nio.Buffer;
import java.text.DecimalFormat;
import javax.swing.Box;
import javax.swing.ImageIcon;
import javax.swing.JButton;
import javax.swing.JFrame;
import javax.swing.JLabel;
import javax.swing.JMenu;
import javax.swing.Timer;
import javax.swing.JMenuBar;
import javax.swing.JMenuItem;
import javax.swing.JPanel;
import javax.swing.JScrollBar;
import javax.swing.JSeparator;
import javax.swing.JToolBar;
import javax.swing.SwingConstants;
import javax.swing.UIManager;
import javax.swing.UnsupportedLookAndFeelException;
import javax.swing.border.EmptyBorder;


// TODO: Auto-generated Javadoc
/**
 * The Class SimulationView.
 */
public class SimulationView extends JPanel{

	
    private JButton playButton;
	private JButton pauseButton;
	private JButton stopButton;	
	private JButton refreshButton;
	private JLabel displayTimer;

    private byte centiseconds = 0;
    private byte seconds = 0;
    private short minutes = 0;
    private DecimalFormat timeFormatter;//timer display format
    private Timer simulationTimer;
    
	
	/** The frame. */
	public ApplicationFrame frame;
	
	/**
	 * Instantiates a new simulation view.
	 *
	 * @param frame
	 *            the frame
	 */
	public SimulationView(ApplicationFrame frame){
		this.frame = frame;
		this.setLayout(new BorderLayout());
		JMenuBar menuBarTop = new JMenuBar();
		JMenu fileMenu = new JMenu("File");
	    menuBarTop.add(fileMenu);
	    JMenu editMenu = new JMenu("Edit");
	    menuBarTop.add(editMenu);
	    
	    frame.setJMenuBar(menuBarTop);
	    
	    JMenuItem newMap = new JMenuItem("New");
	    JMenuItem openMap = new JMenuItem("Open");
	    //openMap.addActionListener(new OpenListener());
	    JMenuItem exitMainMenu = new JMenuItem("Main Menu");
	    exitMainMenu.addActionListener(new MainMenuListener());
	    JMenuItem exit = new JMenuItem("Exit");
	    exit.addActionListener(new ExitListener());
        JMenuItem saveMap = new JMenuItem("Save");
        JMenuItem deleteMap = new JMenuItem("Delete");
        JMenuItem clearMap = new JMenuItem("Clear");
        fileMenu.add(newMap);
        fileMenu.add(openMap);
        fileMenu.addSeparator();
        fileMenu.add(exitMainMenu);
        fileMenu.add(exit);
        editMenu.add(saveMap);
        editMenu.add(clearMap);
        editMenu.add(deleteMap);
     
        //-----------------------------timer implementation
		JMenuBar menubar = new JMenuBar();
	    this.add(menubar, BorderLayout.NORTH);
	    
	    ImageIcon playImage = new ImageIcon("./images/play.png");
		ImageIcon stopImage = new ImageIcon("./images/stop.png");
		ImageIcon pauseImage = new ImageIcon("./images/pause.png");
		ImageIcon refreshImage = new ImageIcon("./images/refresh.jpg");
		
		playButton = new JButton(playImage);
		pauseButton = new JButton(pauseImage);
		stopButton = new JButton(stopImage);	
		refreshButton = new JButton(refreshImage);
		displayTimer = new JLabel("Display Timer");
	    JLabel timerLabel = new JLabel("Timer: ");
	   // simulationTimer = new Timer();
	    timeFormatter = new DecimalFormat("00");
	    
        
	playButton.addActionListener(new PlayListener());
	pauseButton.addActionListener(new PauseListener());
	stopButton.addActionListener(new StopListener());
	refreshButton.addActionListener(new RefreshListener());
	simulationTimer = new Timer(10, new ActionListener() {
         @Override
         public void actionPerformed(ActionEvent e) {
         	if(centiseconds<99){	
         		centiseconds++;
         	} else {
         		if(centiseconds==99){
         			seconds++;
         			centiseconds=0; 
         		}else if(seconds<60){
         				seconds++;
         			} else {
         				if(seconds==60){
         				minutes++;
         				seconds=0;
         			}
         		}
         	}
             displayTimer.setText(timeFormatter.format(minutes) + ":"
                     + timeFormatter.format(seconds) + "."
                     + timeFormatter.format(centiseconds));
         }
     });

	displayTimer.setText(timeFormatter.format(minutes) + ":"
             + timeFormatter.format(seconds) + "."
             + timeFormatter.format(centiseconds));

	
		
		//space between buttons
		menubar.add(timerLabel);
		menubar.add(Box.createRigidArea(new Dimension(10,0)));
		menubar.add(displayTimer);
		menubar.add(Box.createRigidArea(new Dimension(20,0)));
		menubar.add(playButton);
		menubar.add(Box.createRigidArea(new Dimension(10,0)));
		menubar.add(pauseButton); 
		menubar.add(Box.createRigidArea(new Dimension(10,0)));
		menubar.add(stopButton);
		menubar.add(Box.createRigidArea(new Dimension(10,0)));
		menubar.add(refreshButton);  
		
		
	
		/*JButton mainMenu = new JButton("Main Menu");
		mainMenu.addActionListener(new MainMenuListener());
		menubar.add(mainMenu);*/
		
		JToolBar toolbar = new JToolBar(JToolBar.VERTICAL);
		toolbar.setFloatable(false);
		this.add(toolbar, BorderLayout.EAST);
		
		JButton selectMapButton = new JButton("Open Sets of Maps");
		//selectMapButton.addActionListener(new OpenListener());
		selectMapButton.setAlignmentX(Component.CENTER_ALIGNMENT);
		toolbar.add(selectMapButton);
		//selectMapButton.setBackground(new Color(100,100,204));
		Font font = new Font("Avenir Book", Font.BOLD, 14);
		selectMapButton.setFont(font);
		selectMapButton.setForeground(Color.BLUE);
		toolbar.setPreferredSize(new Dimension(200,200));
		toolbar.add(selectMapButton);
		toolbar.addSeparator(new Dimension(100,150));
		
		JLabel carLabel= new JLabel("Select Number of Cars");
		carLabel.setFont(font);
		carLabel.setAlignmentX(Component.CENTER_ALIGNMENT);
		toolbar.add(carLabel);
		
		// 0 to n. n will be automatically calculated
		//TODO SET TO MAX
		JScrollBar carScrollBar = new JScrollBar(JScrollBar.HORIZONTAL, 1, 10, 0, 100); 
		carScrollBar.setAlignmentY(Component.TOP_ALIGNMENT);
		carScrollBar.setMinimum(1);
		carScrollBar.setMaximum(100);
		JLabel limitsLabel = new JLabel();
		limitsLabel.setText("" +carScrollBar.getMinimum() + "                              " + carScrollBar.getMaximum());
		limitsLabel.setAlignmentX(Component.CENTER_ALIGNMENT);
		toolbar.setPreferredSize(new Dimension(200,200));
		toolbar.add(carScrollBar);
		toolbar.add(limitsLabel);
		toolbar.addSeparator(new Dimension(100,150));
	
		
		JLabel motionLabel= new JLabel("Select Car Speed"); 
		motionLabel.setFont(font);
		motionLabel.setAlignmentX(Component.CENTER_ALIGNMENT);
		
		toolbar.add(motionLabel);
		JScrollBar motionScrollBar = new JScrollBar(JScrollBar.HORIZONTAL,1, 1, 0, 10);
		motionScrollBar.setAlignmentY(Component.TOP_ALIGNMENT);
		motionScrollBar.setMinimum(1);
		motionScrollBar.setMaximum(10);
		JLabel mLabel = new JLabel();
		mLabel.setText("" +motionScrollBar.getMinimum() + "                               " + motionScrollBar.getMaximum());
		mLabel.setAlignmentX(Component.CENTER_ALIGNMENT);
		
		toolbar.setPreferredSize(new Dimension(200,200));
		toolbar.add(motionScrollBar);
		toolbar.add(mLabel);
		toolbar.addSeparator(new Dimension(100,150)); //(new JSeparator(SwingConstants.HORIZONTAL));

	}
	
	
	
	/**
	 * The listener interface for receiving mainMenu events. The class that is
	 * interested in processing a mainMenu event implements this interface, and
	 * the object created with that class is registered with a component using
	 * the component's <code>addMainMenuListener<code> method. When
	 * the mainMenu event occurs, that object's appropriate
	 * method is invoked.
	 *
	 * @see MainMenuEvent
	 */

	
	public class MainMenuListener implements ActionListener{
		
		/* (non-Javadoc)
		 * @see java.awt.event.ActionListener#actionPerformed(java.awt.event.ActionEvent)
		 */
		public void actionPerformed(ActionEvent arg0){
			frame.removeView();
			frame.addView(new MainView(frame));
		}
	}
	
	//listener for exit
	public class ExitListener implements ActionListener{	
		/* (non-Javadoc)
		 * @see java.awt.event.ActionListener#actionPerformed(java.awt.event.ActionEvent)
		 */
		public void actionPerformed(ActionEvent arg0){
			System.exit(0);
		}
	}
	
	
	/*public class OpenListener implements ActionListener{
		public void actionPerformed(ActionEvent arg0){
			MapChoiceView openMap = new MapChoiceView();
		}
	}
	
	public class MapListener implements ActionListener{
		public void actionPerformed(ActionEvent arg0){
			MapChoiceView selectMapButton = new MapChoiceView();
			//Load set of predefined maps
		}
	}*/

	public class PlayListener implements ActionListener{
		public void actionPerformed(ActionEvent arg0) {
			simulationTimer.start();
		}
	}
	
	public class PauseListener implements ActionListener{
		public void actionPerformed(ActionEvent arg0){
			simulationTimer.stop();
			
		}
	}
	
	public class StopListener implements ActionListener{
		public void actionPerformed(ActionEvent arg0){
			simulationTimer.stop();
		}
	}
	
	public class RefreshListener implements ActionListener{
		public void actionPerformed(ActionEvent arg0){
			 simulationTimer.stop();

             centiseconds = 0;
             seconds = 0;
             minutes = 0;

             displayTimer.setText(timeFormatter.format(minutes) + ":"
                     + timeFormatter.format(seconds) + "."
                     + timeFormatter.format(centiseconds));
		}
	}
	
	/*public class TimerListener implements ActionListener{
        @Override
        public void actionPerformed(ActionEvent e) {
        	if(centiseconds<99){	
        		centiseconds++;
        	} else {
        		if(centiseconds==99){
        			seconds++;
        			centiseconds=0; 
        		}else if(seconds<60){
        				seconds++;
        			} else {
        				if(seconds==60){
        				minutes++;
        				seconds=0;
        			}
        		}
        	}
        	displayTimer.setText(timeFormatter.format(minutes) + ":"
                    + timeFormatter.format(seconds) + "."
                    + timeFormatter.format(centiseconds));
        }
    }*/
	
	public class IncreaseNumberOfCars implements ActionListener{
		public void actionPerformed(ActionEvent arg0){
			//TODO Implement listener for increase number of cars
		
	}
}
	public class IncreaseCarSpeed implements ActionListener{
		public void actionPerformed(ActionEvent arg0){
			//TODO Implement listener for increase car speed 
		}
	}
}
\end{lstlisting}


MainView.java

\begin{lstlisting}[numbers=left, numberstyle=\small, numbersep=8pt,  framexleftmargin=1pt, framexrightmargin=10pt ]

/*
 * @author  Maxim Vasilishin
 * @version 1.0
 */
package gui;

import java.awt.Component;
import java.awt.event.ActionEvent;
import java.awt.event.ActionListener;

import javax.swing.BoxLayout;
import javax.swing.JButton;
import javax.swing.JPanel;

// TODO: Auto-generated Javadoc
/**
 * The Class MainWindow.
 */
public class MainView extends JPanel {
	
	/**
	 * Instantiates a new main window.
	 */
	
	private ApplicationFrame frame;
	
	/**
	 * Instantiates a new main view.
	 *
	 * @param frame
	 *            the frame
	 */
	public MainView (ApplicationFrame frame){
		this.frame = frame;
		 BoxLayout layout = new BoxLayout(this, BoxLayout.Y_AXIS);
		 this.setLayout(layout);
		 JButton button = new JButton("Simulation");
	     button.setAlignmentX(Component.CENTER_ALIGNMENT);
	     this.add(button);
	     JButton button1 = new JButton("Map Builder");
	     button1.setAlignmentX(Component.CENTER_ALIGNMENT);
	     this.add(button1);
	     JButton button2 = new JButton("Exit");
	     button2.setAlignmentX(Component.CENTER_ALIGNMENT);
	     this.add(button2);
	     
	     button2.addActionListener(new ExitListener());
		 button.addActionListener(new SimulationListener());
		 button1.addActionListener(new MapBuilderListener());
		 
		
	}
	
	//Functionality of SimulationButton-New Window (Main.java)
			/**
	 * The listener interface for receiving simulation events. The class that is
	 * interested in processing a simulation event implements this interface,
	 * and the object created with that class is registered with a component
	 * using the component's <code>addSimulationListener<code> method. When
	 * the simulation event occurs, that object's appropriate
	 * method is invoked.
	 *
	 * @see SimulationEvent
	 */
public class SimulationListener implements ActionListener{
				
				/* (non-Javadoc)
				 * @see java.awt.event.ActionListener#actionPerformed(java.awt.event.ActionEvent)
				 */
				public void actionPerformed(ActionEvent arg0){
					frame.removeView();
					frame.addView(new SimulationView(frame));
				}
			}
			
			//Functionality of MapBuilderButton-New Window (SystemPrint.java)
			/**
			 * The listener interface for receiving mapBuilder events. The class
			 * that is interested in processing a mapBuilder event implements
			 * this interface, and the object created with that class is
			 * registered with a component using the component's
			 * <code>addMapBuilderListener<code> method. When
			 * the mapBuilder event occurs, that object's appropriate
			 * method is invoked.
			 *
			 * @see MapBuilderEvent
			 */
public class MapBuilderListener implements ActionListener{
				
/* (non-Javadoc)
* @see java.awt.event.ActionListener#actionPerformed(java.awt.event.ActionEvent)
*/
public void actionPerformed(ActionEvent arg0){
		frame.removeView();
		frame.addView(new RoadEditorView(frame));
					
				}
			}
			
//Functionality of ExitButton
/**			 * The listener interface for receiving exit events. The class that
			 * is interested in processing a exit event implements this
			 * interface, and the object created with that class is registered
			 * with a component using the component's
			 * <code>addExitListener<code> method. When
			 * the exit event occurs, that object's appropriate
			 * method is invoked.
			 *
			 * @see ExitEvent
			 */
public class ExitListener implements ActionListener{
				
/* (non-Javadoc)
* @see java.awt.event.ActionListener#actionPerformed(java.awt.event.ActionEvent)
*/
public void actionPerformed(ActionEvent arg0){
					System.exit(0);
				}
			}
}



\end{lstlisting}
